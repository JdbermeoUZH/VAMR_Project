\begin{enumerate}
    \item \textbf{Intialization Module}
    
    \begin{enumerate}
        \item Define/Specify the logic or argument to pick specific initialization frames for each dataset.
        
        \item Identify features in both frames with one of the following techniques:
            \begin{itemize}
                \item Corner detection: Harris, Shi-Tomasi, FastCorner
                \item Sift Features
                \item \textbf{Bonus}: Lift
            \end{itemize}
        
        \item Match the keypoints between frames to find keypoint correspondences between different views (between frames or between multiple cameras).
            \begin{itemize}
                \item Choose a type of distance/similarity metric
                \item Choose to do matching with either pairwise comparison or KLT 
            \end{itemize}
        
        \item Estimate the point cloud of the landmarks.
        
        \item Estimate relative pose of the next frame by calculating the fundamental matrix.
            \begin{itemize}
                \item Use 8 point ransac (estimateFundamentalMatrix)
                \item \textbf{Bonus}: 5 point ransac
                \item \textbf{Bonus}: 1 point ransac + 5 point ransac
            \end{itemize}
        
        \item Refine the estimated pose with Bundle Correction. 
        
    \end{enumerate}
    
    \item \textbf{Continous VO Module}
    \begin{enumerate}
        \item Create functions that iterates over all subsequent states, always using the keypoints in frame $i$ and compare/match them to those of frame $i+1$ to estimate the pose of frame $i+1$. (practically following the same pipeline as initialization).
        
        \item Persist the the estimated poses at each step.
        
        \item Choose error metrics and create function that calculates global and local error in pose estimation:
            \begin{itemize}
                \item Figure out how to obtain the ground truth of poses at the local and global level
                \item Create function that computes the metric for both
            \end{itemize}
        
        \item \textbf{Bonus:} Use some method to correct global drift.
        
    \item \textbf{Visualizing results}:
        \item Create interface that runs the pipeline 
            \begin{itemize}
                \item Display current image, plot's current and past features and the distance between them (plots matched features). 
                \item Plot trajectory and landmarks of the last 20 frames.
                \item Plot \# of tracked landmarks in the last 20 frames 
                \item Plots full trajectory
                \item \textbf{Bonus:} Plots trajectory over sattelite map
            \end{itemize}
            
        \item As the script runs, stitch the resulting matched features into a video (screencast)
        
    \end{enumerate}
    
\end{enumerate}

